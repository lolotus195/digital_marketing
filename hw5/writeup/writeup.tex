\documentclass[11pt, fleqn]{article}

\usepackage[usenames,dvipsnames,svgnames,table]{xcolor}
\usepackage{amsmath}
\usepackage{amsfonts}
\usepackage[margin=1in]{geometry} % To set the margin widths
\usepackage{graphicx}
\usepackage{listings}
\usepackage{multirow}
\usepackage{tabularx}
\usepackage{varioref}
\usepackage[noabbrev,capitalize]{cleveref}
\usepackage[group-separator={,}]{siunitx}
\usepackage{subcaption}
\usepackage{titlesec}
\usepackage{lscape}
\usepackage{bm}
\usepackage{chngpage}
\usepackage[titletoc,toc,title]{appendix}

\renewcommand\thesection{\arabic{section}}
\renewcommand\thesubsection{\thesection\alph{subsection}}
\crefname{appsec}{Appendix}{Appendices}

\lstset{
  frame=single,
  basicstyle=\ttfamily,% print whole listing small
  language=R,
  aboveskip=3mm,
  belowskip=3mm,
  showstringspaces=false,
  columns=flexible,
  numbers=none,
  commentstyle=\color{ForestGreen},
  stringstyle=\color{Maroon},
  breaklines=true,
  breakatwhitespace=true,
  tabsize=2,
  literate={<-}{{$\gets$}}1 {~}{{$\sim$}}1
}

\sisetup{output-exponent-marker=\textsc{e}}

\setlength{\parskip}{12pt} % Sets a blank line in between paragraphs
\setlength\parindent{0pt} % Sets the indent for each paragraph to zero

\newcommand\singlefig[2] {
  \begin{figure}[htb]
    \centering
    \caption{#2}
    \includegraphics[scale=.5]{#1.pdf}
    \label{fig:#1}
  \end{figure}  
}
\newcommand\doublefig[5] {
  \begin{figure}[htb]
    \centering
    \caption{#1}
    \begin{subfigure}[b]{0.49\textwidth}
      \caption{#3}
      \includegraphics[width=\textwidth]{#2.pdf}
      \label{fig:#2}
    \end{subfigure}
    \hfill
    \begin{subfigure}[b]{0.49\textwidth}
      \caption{#5}
      \includegraphics[width=\textwidth]{#4.pdf}
      \label{fig:#4}
    \end{subfigure}
  \end{figure}
}


\begin{document}

\title{Homework \#5\\
Digital and Algorithmic Marketing (37304-01)}
\author{
Brian Chingono, Will Clark, Matthew DeLio, Jonathan Stevens (\textbf{Group \#8})\\
University of Chicago Booth School of Business}

\maketitle

\section{} \label{sec:q1} % Question 1
To identify the optimal uniform price that ZipRecruiter should charge, we estimated the following regression:
\[ s_i = \alpha + P_i \beta + \varepsilon_i \]
where $s_i$ is a binary indicator for whether or not customer $i$ subscribed and $P_i$ is the price in dollars that the customer was charged. This regression is our estimated demand function for ZipRecruiter's services across the range of tested prices. We show the results of this regression in \vref{fig:q1_fit}---the blue line is the estimated demand curve, which is plotted against the empirically observed demand schedule in red.

\singlefig{q1_fit}{Actual/Fitted Subscription Rate}

We can then use this estimated demand function to compute the company's profit as a function of the price charged:
\[ \Pi_i = s_i \left( P_i - c \right) \]
where $c$ is the cost of servicing a customer (assumed to be constant across all customers). We show the profit function for varying levels of cost per customer in \vref{fig:q1_profits}. The optimal price is the price at which the profit function is maximized. For a cost per customer of \$10, the optimal price is \$303 (shown in the dark yellow line in \cref{fig:q1_profits}). We can observe in this plot as well that the optimal price must be raised as the cost to service a customer increases.

\singlefig{q1_profits}{Profits Under Uniform Pricing (with Varying Marginal Costs)}

\section{} \label{sec:q2} % Question 2
To begin our consideration on how to optimally segment the market we fit two regressions to the data:
\begin{align}
s_i &= \alpha +  P_i \beta_P + L_i \beta_L + P_i L_i \beta_{P,L} + \varepsilon_i \label{eq:regr_state} \\
s_i &= \alpha +  P_i \beta_P + JC_i \beta_L + P_i JC_i \beta_{P,JC} + \varepsilon_i \label{eq:regr_cat}
\end{align}
In the \vref{eq:regr_state}, we fit the subscription rate ($s_i$) on price ($P_i$) interacted with job location ($L_i$). This allows us to compute a state-specific price elasticity. We immediately find that two states do not have enough data to even estimate an elasticity: GU (Guam) and NT (the Northern Territories in Canada).  Additionally, we found that for another 25 states, we could not reject the null hypothesis of positive price elasticity at even a 50\% confidence level. For all 27 of these states, we apply the optimum pricing strategy calculated in \cref{sec:q1} instead of trying to predict an optimal price with flawed data.\footnote{We believed that in no case is elasticity actually positive, so if the data tell us it is, we do not believe the data.} Note that we incorporate the size of the segment into the total profits calculation.  For the remaining 34 states that have significant and negative elasticities we use the optimization function in R to compute a price that optimizes the profit -- or, in this case with no marginal cost, revenue.

The same exercise is performed using job category instead of location as a segmentation approach \vref{eq:regr_cat}.  In this case there are far fewer cases, only 13, where we do not have good elasticity measurements.

Using both models we calculate optimal price and the maximum profit per segment.  \Vref{fig:seg_mc_0} shows the breakdown of price vs. segment size. The segments are ordered from most to least profitable; the segment width is proportional to the number of subscribing customers at the optimal price, and the height of each segment is the optimal price (so that the area of each segment is the optimal profit).\footnote{Additionally, those segments for which there was insufficient data to estimate a segment-specific model are plotted in blue rather than in red.} We can see that when we segment by job category rather than by state, there is much more variation in optimal price and the top segments are much more profitable. This observation is corroborated by the listing of the most profitable segments under each strategy, listed in \vref{tab:summary_mc_0}.

Additionally \vref{fig:q3_predicted} shows the fitted segment-specific profit functions. It makes clear that there are some advantages of charging customers different prices by segment because there is such wide variation in the profitability of each segment.  Similarly, the distribution of optimal prices in \vref{fig:q2_hist} confirms that there is enough variability that segmenting by either state or by job type should increase ZipRecruiter's overall profitability.

Finally, to recommend which segment type to use, we sum the profits from each segmentation strategy to form the expected profits (along with a 95\% confidence interval).  \Vref{fig:q3_profits_summary} shows the optimized profit by segmentation type.  Here we find that for a marginal cost of \$0, that segmenting by \textbf{Job Category} is more optimal than segmenting by Job Location or not segmenting at all.  This yields \$517,825 in revenue/profits vs \$461,564 and \$481,084 for state-based segmenting and uniform pricing, respectively.\footnote{Note that the confidence intervals for the segmentation types indicate that Job Category is not significantly different from Job Location and not segmenting.}

\singlefig{seg_mc_0}{Segment-Specific Revenue with Cost/Customer = \$0}
\singlefig{q3_predicted}{Profit Curves by Segment and Marginal Cost}
\singlefig{q2_hist}{Histogram of Optimal Prices by Segment and Marginal Cost}

% latex table generated in R 3.3.0 by xtable 1.8-2 package
% Tue May 17 21:11:56 2016
\begin{table}[ht]
\centering
\caption{Top 10 Segments for Cost = \$0} 
\label{tab:summary_mc_0}
\begin{tabular}{llrllr}
  \hline
State & Revenue & Price & Job Category & Revenue & Price \\ 
  \hline
CA & 98,331 & 349 & Sales + Biz Dev & 132,855 & 553 \\ 
  FL & 50,566 & 341 & Healthcare & 59,701 & 288 \\ 
  TX & 36,362 & 294 & Information Technology & 55,222 & 349 \\ 
  NY & 29,455 & 232 & Construction/Skilled Trade & 39,812 & 251 \\ 
  NJ & 18,810 & 313 & Accounting/Finance & 37,923 & 301 \\ 
  OH & 15,315 & 229 & Customer Service & 26,073 & 610 \\ 
  IL & 15,041 & 172 & Admin/Secretarial & 17,184 & 171 \\ 
  GA & 14,112 & 294 & Manufacturing/Operations & 16,291 & 273 \\ 
  NC & 12,308 & 294 & Trucking/Transport & 14,691 & 316 \\ 
  MD & 10,761 & 244 & Engineering & 13,881 & 272 \\ 
   \hline
\end{tabular}
\end{table}


\section{} % Question 3
For this question, we perform the same exercise as was performed in \cref{sec:q2} except that we added a marginal cost of \$10 to the profit function.  As before we find the segment-specific revenue (\vref{fig:seg_mc_10}), the profit curves (\vref{fig:q3_predicted}), histogram of optimal prices (\vref{fig:q2_hist}), and top-10 profit contributions by segment (\vref{tab:summary_mc_10}).  One thing to note from the histogram, is that it shows the expected increase in the optimal prices with increasing marginal cost (note: this is a very subtle effect; the marginal cost is relatively small compared to the reasonable prices to charge).

As before, we summarize our findings by examining the overall profits.  As before we turn to \vref{fig:q3_profits_summary} and note that \textbf{Job Category} again appears to have better profits than Job Location and not segmenting, but, again, the confidence intervals do not show much, if any, differentiation between the three schemes.

\singlefig{seg_mc_10}{Segment-Specific Revenue with Cost/Customer = \$10}
\singlefigscale{q3_profits_summary}{Overall Profits by Segmentation Type and Marginal Cost}{0.8}
% latex table generated in R 3.3.0 by xtable 1.8-2 package
% Tue May 17 21:11:56 2016
\begin{table}[ht]
\centering
\caption{Top 10 Segments for Cost = \$0} 
\label{tab:summary_mc_10}
\begin{tabular}{llrllr}
  \hline
State & Revenue & Price & Job Category & Revenue & Price \\ 
  \hline
CA & 95,544 & 357 & Sales + Biz Dev & 130,473 & 562 \\ 
  FL & 49,101 & 349 & Healthcare & 57,662 & 297 \\ 
  TX & 35,143 & 303 & Information Technology & 53,660 & 358 \\ 
  NY & 28,209 & 241 & Construction/Skilled Trade & 38,252 & 259 \\ 
  NJ & 18,217 & 322 & Accounting/Finance & 36,681 & 309 \\ 
  OH & 14,658 & 237 & Customer Service & 25,648 & 619 \\ 
  IL & 14,189 & 181 & Admin/Secretarial & 16,202 & 179 \\ 
  GA & 13,639 & 303 & Manufacturing/Operations & 15,703 & 281 \\ 
  NC & 11,895 & 303 & Trucking/Transport & 14,233 & 325 \\ 
  MD & 10,327 & 252 & Engineering & 13,378 & 280 \\ 
   \hline
\end{tabular}
\end{table}


\end{document}

% \input{.tex}

% \begin{figure}[!htb]
%   \centering
%   \caption{}
%   \begin{subfigure}[b]{0.49\textwidth}
%     \caption{}
%     \includegraphics[width=\textwidth]{.pdf}
%     \label{fig:}
%   \end{subfigure}
%   \hfill
%   \begin{subfigure}[b]{0.49\textwidth}
%     \caption{}
%     \includegraphics[width=\textwidth]{.pdf}
%     \label{fig:}
%   \end{subfigure}
% \end{figure}

% \begin{figure}[!htb]
%   \centering
%   \caption{}
%   \includegraphics[scale=.5]{.pdf}
%   \label{fig:}
% \end{figure}