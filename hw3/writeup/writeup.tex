\documentclass[11pt, fleqn]{article}

\usepackage[usenames,dvipsnames,svgnames,table]{xcolor}
\usepackage{amsmath}
\usepackage{amsfonts}
\usepackage[margin=1in]{geometry} % To set the margin widths
\usepackage{graphicx}
\usepackage{listings}
\usepackage{multirow}
\usepackage{tabularx}
\usepackage{varioref}
\usepackage[noabbrev,capitalize]{cleveref}
\usepackage[group-separator={,}]{siunitx}
\usepackage{subcaption}
\usepackage{titlesec}
\usepackage{lscape}
\usepackage{bm}
\usepackage{chngpage}
\usepackage[titletoc,toc,title]{appendix}

\renewcommand\thesection{\arabic{section}}
\renewcommand\thesubsection{\thesection\alph{subsection}}

\lstset{
  frame=single,
  basicstyle=\ttfamily,% print whole listing small
  language=R,
  aboveskip=3mm,
  belowskip=3mm,
  showstringspaces=false,
  columns=flexible,
  numbers=none,
  commentstyle=\color{ForestGreen},
  stringstyle=\color{Maroon},
  breaklines=true,
  breakatwhitespace=true,
  tabsize=2,
  literate={<-}{{$\gets$}}1 {~}{{$\sim$}}1
}

\sisetup{output-exponent-marker=\textsc{e}}

\setlength{\parskip}{12pt} % Sets a blank line in between paragraphs
\setlength\parindent{0pt} % Sets the indent for each paragraph to zero

\begin{document}

\title{Homework \#3\\
Digital and Algorithmic Marketing (37304-01)}
\author{
Brian Chingono, Will Clark, Matthew DeLio, Jonathan Stevens (\textbf{Group \#8})\\
University of Chicago Booth School of Business}

\maketitle

\section{Gender-Based Preferences of Message Senders}

\vref{fig:heatmap_female,fig:heatmap_male} show, for a given sender rating, the difference between the conditional probability of sending a message (conditioned on the senders rating) and the marginal probability of sending a message across all receiver ratings. 

To take a concrete example from \cref{fig:heatmap_female}: suppose I am a female with a rating of 6. I am 2.2 percentage points more likely to target a man with a rating of 7 than the average woman is. Formally:
\[ Pr(\text{message}|F=7,M=6) - Pr(\text{message}|M=6) = 2.2 \text{ percentage points}\]

\doublefig{}{heatmap_female}{Relative Preferences of Female Senders}{heatmap_male}{Relative Preferences of Male Senders}

Three random observations we can make about these data:
\begin{itemize}
\item In general, women tend to look for men who are rated similarly to themselves, while men tend to look almost exclusively for women who are rated highly (even if the men are not rated highly themselves). The exceptions occur at the extreme end of the spectrum: men who are rated 1 or 2 are slightly more likely to contact women who are similarly rated.
\item For both genders, it looks like there is a penalty for being rated as average. Men who are rated 5 or 6, or women who are rated 7, are almost all conditionally less likely to be messaged than their above- or below-average peers.
\item Every man seems to think he has a chance with a woman who is rated 10 out of 11; these women are all conditionally more likely to be messaged. Women who are rated 11 out of 11, however, are apparently too good for some men at the bottom of the scale (1-4), who are conditionally less interested.
\end{itemize}

We can formalize this last point with a regression:
\[ R_{i,j} = \alpha_j + \beta_j S_{i,j} + \varepsilon_{i,j} \]
where $R_{i,j}$ is the rating of receiver $i$, $S_{i,j}$ is the rating of sender $i$, and $j=\text{\{female,male\}}$. Essentially we are trying to predict the rating of the receiver given the rating of the sender.

The estimated intercept will tell us what is the unconditional expectation for the receivers' ratings. The estimated slope coefficient will tell us the degree to which men or women target those with similar ratings. The higher the coefficient, the more likely it is that senders target similarly rated receivers (i.e. if $\hat{\beta}=1$ then people would only message those who are rated exactly the same as themselves).

% latex table generated in R 3.2.3 by xtable 1.8-2 package
% Mon Apr 25 21:30:19 2016
\begin{table}[ht]
\centering
\caption{Regressing Receiver Looks on Sender Looks} 
\label{tab:receiver_on_sender}
\begin{tabular}{rrrr}
  \hline
 & $\alpha_j$ & $\beta_j$ & R-squared \\ 
  \hline
Female senders & 5.17 & 0.209 & 0.0466 \\ 
  Male senders & 5.94 & 0.159 & 0.0263 \\ 
   \hline
\end{tabular}
\end{table}


For both genders, the unconditional expectation of the receivers' rating is below average (given that the average is a 6). Somewhat surprisingly, women (on average) tend to send messages to men who are slightly \textit{better} looking than they are, while men (on average) send messages to women who are slightly \textit{worse} looking than they are.\footnote{The rating of the average female sender is 5.05 (lower than $\hat{\alpha}_{F}$); the rating of the average male sender is 5.32 (higher than $\hat{\alpha}_{M}$).} Given that the coefficient $\hat{\beta}$ is higher for women than for men, in the aggregate, women are more likely than men to target someone who is rated similarly. Note, however, that the R-squared values for both regressions are very low, so this simple model explains very little of the targeting preferences for either gender.

We can see this most clearly in \vref{fig:dotplot_female,fig:dotplot_male}. The volume of messages for a given receiver/sender pair are plotted (bigger dots $\rightarrow$ more messages). Women tend to be a little more discriminating than men, as the bulk of the messages seem to be slightly more concentrated near the diagonal axis. The regression line for each gender is plotted, and we can see that the line for women is a little bit steeper (though not by much!) than the line for the men.


\doublefig{}{dotplot_female}{Aggregate Sending Behavior of Females}{dotplot_male}{Aggregate Sending Behavior of Males}


\section{}

The utility/preference function does change depending on the looks of the sender, and these differences are unique to each gender.  Our primary findings regarding these preferences are as follows:

We first examined the probability of a male sending a message to women of various looks ratings based on male looks ratings. \vref{fig:Q2_Male_senders_color} shows three categories of males and their respective probabilities of contacting women across the looks rating spectrum.  The green line shows all males, the red line shows only males with above average looks ratings and the blue line shows males of below average looks ratings.  As you can see, the likelihood of males to contact a woman is relatively consistent across all female looks ratings up to 9.  For women of looks rating 10 or 11, there is some divergence as better looking men are more likely to contact them than below-average looking men.  

\singlefigscale{Q2_Male_senders_color}{Graph of male message sending probability based on female recipient looks category}{.6}

We then expanded this analysis to examine the likelihood of men of every looks rating contacting women of every looks rating.  \vref{fig:Q2_Male_heatmap} shows a heatmap of the predicted probability ("yhat") of contact for male senders and female receivers at every looks rating.

\singlefigscale{Q2_Male_heatmap}{Heatmap of male message sending probability based on female recipient looks category}{.6}

After analyzing the likelihood of men of different looks ratings contacting women of different looks ratings, we then performed the same analysis for women. \vref{fig:Female_looks_color} shows three categories of women and their respective probabilities of contacting men across the looks rating spectrum. The blue line shows all women, the red line shows only women with below average looks ratings and the green line shows women of above average looks ratings.  The probability of women in all categories contacting men of different looks categories has a greater variance than for men.  Women of high looks ratings are very unlikely to contact men of low looks ratings and much more likely to contact men of high looks ratings than women of below-average looks ratings.  Conversely, women of below average looks ratings are much more likely to contact men of low looks ratings and much less likely to contact men of high looks ratings.  

\singlefigscale{Female_looks_color}{Graph of female message sending probability based on male recipient looks category}{.6}

We then expanded this analysis to examine the likelihood of women of every looks rating contacting men of every looks rating.  \vref{fig:Q2_Female_heatmap} shows a heatmap of the predicted probability ("yhat") of contact for male senders and female receivers at every looks rating.

\singlefigscale{Q2_Female_heatmap}{Heatmap of female message sending probability based on male recipient looks category}{.6}

\section{} \label{sec:q3}
We created an 11 x 11 matrix with the match scores that are associated with different combinations of male looks and female looks. \vref{fig:q3_score_heatmap} shows a heatmap of the results. The red cells indicate low match scores and the green cells represent high match scores. 

Across the entire matrix, the average match score is 0.088. Overall, the following observations can be made from \cref{fig:q3_score_heatmap}:
\begin{itemize}
\item The most attractive women (categories "11" and "10") have above-average match scores, regardless of the appearance of their male counterpart. For example, the match scores of women in the "11" category range from 0.10 to 0.14. The match scores of women in the "10" category range from 0.10 to 0.13.

\item On the other hand, the least attractive women (categories "1" and "2") have below-average match scores, regardless of the appearance of their male counterpart. Specifically, the  match scores of women in the "1" category range from 0.05 to 0.07. The match scores of women in the "2" category range from 0.06 to 0.08.

\item These two points support our theory that men are particularly sensitive to the physical attractiveness of women. If we only condition on receiver looks, it appears that men typically want to send messages to attractive women, but they do not want to contact unattractive women.

\item The pattern seen in \cref{fig:q3_score_heatmap} seems to be driven by mens' sensitivity to womens' attractiveness. Across the range of male attractiveness ("1" to "11"), the match scores are below average when men are paired with unattractive women (categories "1" and "2"). However, the match scores are all above-average when men are paired with highly attractive women (categories "10" and "11").

\item The match score function is \textit{pred.match("Male looks", "Female looks")}. Since men are more sensitive to womens' appearance, we would expect $pred.match(2,10)$ to be greater than $pred.match(10,2)$. Indeed, as we can see from the matrix in \cref{fig:q3_score_heatmap}, $pred.match(2,10) = 0.10$ and $pred.match(10,2) = 0.07$. As expected, $0.10 > 0.07$.

\end{itemize}

\singlefigscale{q3_score_heatmap}{Heatmap of Match Scores by Gender Looks (Incorporating Receiver Looks Only)}{.6}

\section{}
First we regressed the binary contact indicator, \textit{y}, on \textit{ReceiverLooks}, \textit{SenderLooks}, and \textit{SenderGender} allowing those terms to interact with each other (see \vref{lst:q4_all}).  This took quite a while to run -- about 3 minutes on my 2010-era notebook -- and generated 242 coefficients ($=11 * 11 * 2$).  While many of the coefficients were insignificant, they gave us a model we could use to account for sender/receiver looks as well as gender differences.  With this model, we then calculated a match score (see \vref{lst:q4_pred}) which we then used to generate a heat-map.  In words the heat map in \vref{fig:q4_heat} shows
\begin{itemize}
\item matches between very different-looking people tend to have below-average match-scores;
\item matches between similar-looking people (both with above average looks) tend to have above-average match-scores;
\item matches between people with average, to below-average looks tends to be a mixed-bag.
\end{itemize}

Overall, these results contrast sharply with those in \vref{sec:q3}, which indicates that the mere involvement of a person with a high looks score is enough to yield a high match-score.

\singlefigscale{q4_heat}{Heatmap of Match Scores by Gender Looks (Incorporating Sender \& Receiver Looks)}{.6}

\begin{lstlisting}[language=R,label=lst:q4_all,caption=Regressing y on all available data.]
glm(y ~ ReceiverLooks * SenderLooks * SenderGender, data=df, family="binomial")
\end{lstlisting}

\lstinputlisting[language=R,label=lst:q4_pred,caption=Predicting scores using all data.]{predict_snippet.R}

%%%%%%%%%%% BEGIN APPENDIX
\clearpage
\begin{appendices}

\crefalias{section}{appsec}
\section{Data Exploration} \label{app:1}
To begin our analysis we took a cursory look at the available data to see where we might be lacking the ability to make predictions.  In general females make up just 27.5\% of the data and make contact at a rate of 8.1\% (with males making contact at 9.6\%).  On the surface, these results aren't disheartening, but we next turn our concern to the variety of match data.  For example, we would be a little concerned if we never have a 11 female matched with a 1 male, or vice-versa. \Vref{fig:q0_data_heat} shows  a heatmap of the available information broken down not only by sender gender, but also the looks of the sender/receiver.  In general it shows that we do have available information for all segments.  However, there is a general lack of matches between people of either gender rated 10/11 as well as general lack of matches at extremes (around the 11/1 pairings).

\singlefigscale{q0_data_heat}{Heatmap of available match data}{0.6}
% \singlefigscale{q0_contact_heat}{Heatmap of available contact data}{0.6}


\end{appendices}

\end{document}

% \input{.tex}

% \begin{figure}[!htb]
%   \centering
%   \caption{}
%   \begin{subfigure}[b]{0.49\textwidth}
%     \caption{}
%     \includegraphics[width=\textwidth]{.pdf}
%     \label{fig:}
%   \end{subfigure}
%   \hfill
%   \begin{subfigure}[b]{0.49\textwidth}
%     \caption{}
%     \includegraphics[width=\textwidth]{.pdf}
%     \label{fig:}
%   \end{subfigure}
% \end{figure}

% \begin{figure}[!htb]
%   \centering
%   \caption{}
%   \includegraphics[scale=.5]{.pdf}
%   \label{fig:}
% \end{figure}