\documentclass[11pt, fleqn]{article}

\usepackage[usenames,dvipsnames,svgnames,table]{xcolor}
\usepackage{amsmath}
\usepackage{amsfonts}
\usepackage[margin=1in]{geometry} % To set the margin widths
\usepackage{graphicx}
\usepackage{listings}
\usepackage{multirow}
\usepackage{tabularx}
\usepackage{varioref}
\usepackage[noabbrev,capitalize]{cleveref}
\usepackage[group-separator={,}]{siunitx}
\usepackage{subcaption}
\usepackage{titlesec}
\usepackage{lscape}
\usepackage{bm}
\usepackage{chngpage}
\usepackage[titletoc,toc,title]{appendix}

\renewcommand\thesection{\arabic{section}}
\renewcommand\thesubsection{\thesection\alph{subsection}}
\crefname{appsec}{Appendix}{Appendices}

\lstset{
  frame=single,
  basicstyle=\ttfamily,% print whole listing small
  language=R,
  aboveskip=3mm,
  belowskip=3mm,
  showstringspaces=false,
  columns=flexible,
  numbers=none,
  commentstyle=\color{ForestGreen},
  stringstyle=\color{Maroon},
  breaklines=true,
  breakatwhitespace=true,
  tabsize=2,
  literate={<-}{{$\gets$}}1 {~}{{$\sim$}}1
}

\sisetup{output-exponent-marker=\textsc{e}}

\setlength{\parskip}{12pt} % Sets a blank line in between paragraphs
\setlength\parindent{0pt} % Sets the indent for each paragraph to zero

\newcommand\singlefig[2] {
  \begin{figure}[htb]
    \centering
    \caption{#2}
    \includegraphics[scale=.5]{#1.pdf}
    \label{fig:#1}
  \end{figure}  
}
\newcommand\doublefig[5] {
  \begin{figure}[htb]
    \centering
    \caption{#1}
    \begin{subfigure}[b]{0.49\textwidth}
      \caption{#3}
      \includegraphics[width=\textwidth]{#2.pdf}
      \label{fig:#2}
    \end{subfigure}
    \hfill
    \begin{subfigure}[b]{0.49\textwidth}
      \caption{#5}
      \includegraphics[width=\textwidth]{#4.pdf}
      \label{fig:#4}
    \end{subfigure}
  \end{figure}
}


\begin{document}

\title{Homework \#1\\
Digital and Algorithmic Marketing (37304-01)}
\author{Brian Chingono, Will Clark, Matthew DeLio, Yoni Sarason, Jonathan Stevens\\
University of Chicago Booth School of Business}

\maketitle

\section{} % Question 1

\subsection{K-Means Matching}

Yes. We would use the new DMP. A regression of spend against the explanatory variables shows that the new ecom\_index variable is statistically significant. Moreover, the retail\_index variable is no longer statistically significant after we control for ecom\_index. Therefore, the new DMP provides a variable that allows us to obtain better matches than our original method.

As shown in the chart below, we can obtain a higher maximum profit under the new DMP than under our original method. Under the new DMP, profit is maximized at \$4,413.61 whereas the maximum profit under the old method was \$3,559.14. In both cases, profit is maximized when k-nearest neighbors are set to 3. See \vref{fig:Profits}.

\begin{figure}[!htb]
  \centering
  \caption{Profits vs. Number of Nearest Neighbors}
  \includegraphics[scale=.5]{Profits.pdf}
  \label{fig:Profits}
\end{figure}

\subsection{K-Nearest-Neighbors Matching}

\section{} % Question 2

\end{document}

% \input{.tex}

% \begin{figure}[!htb]
%   \centering
%   \caption{}
%   \begin{subfigure}[b]{0.49\textwidth}
%     \caption{}
%     \includegraphics[width=\textwidth]{.pdf}
%     \label{fig:}
%   \end{subfigure}
%   \hfill
%   \begin{subfigure}[b]{0.49\textwidth}
%     \caption{}
%     \includegraphics[width=\textwidth]{.pdf}
%     \label{fig:}
%   \end{subfigure}
% \end{figure}

% \begin{figure}[!htb]
%   \centering
%   \caption{}
%   \includegraphics[scale=.5]{.pdf}
%   \label{fig:}
% \end{figure}